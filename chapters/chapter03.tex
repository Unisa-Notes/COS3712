\providecommand{\main}{..}
\documentclass[../COS3712_Notes.tex]{subfiles}

\begin{document}
  \setcounter{chapter}{2}
  \chapter{Geometric Objects and Transformations}
    \section{Linear Algebra Notes}
      \subsection{Points, Vectors, and Scalars}
        \begin{description}
          \item[Point] Represented graphically by a single dot.
            A position or location specified by three values placed in parentheses $(x, y, z)$.
            The point $(0, 0, 0)$ represents the point where the three axes intersect
            with the value 0, and is referred to as the \concept{origin}.
            We generally use lowercase letters $p$, $q$ and $r$ to refer to points.
          \item[Vector] Represented graphically by an arrow.
            A direction with a magnitude.
            Also normally written in parentheses, but \emph{does not} specify a position.
            A vector does not have a position.
            We generally use lowercase letters $u$, $v$ and $w$ to refer to vectors.

            Do not confuse a vector with a \concept{line~segment}.
            A line~segment has a begin and endpoint (unlike a vector)
            and does not point in one of two possible directions
            (whereas a vector does).
          \item[Scalar] A value or magnitude.
            Has neither position nor direction.
            We generally use lowercase letters of the Greek alphabet $\alpha$, $\beta$ and $\gamma$
            to represent arbitrary scalars, but sometimes lowercase letters of the alphabet.
        \end{description}

        \begin{sidenote}{Expressing Points and Vectors in Homogeneous Coordinates}
          We can distinguish between points and vectors by expressing them in four dimensions:
          adding a fourth coordinate.
          For a point, the fourth coordinate is 1, and for a vector it is 0.
        \end{sidenote}

      \subsection{Adding and Subtracting Vectors and Points}
        We can add two vectors together, or subtract two vectors from one another.
        We can add a vector to a point (or subtract, which is just adding in the opposite direction).
        We cannot add two points together, but we can subtract one point from another.

        \begin{description}
          \item[Sum of Two Vectors] Produces a \emph{vector}.
            Add the corresponding coordinates.
          \item[Difference Between Two Vectors] Produces a \emph{vector}.
            Subtract the corresponding coordinates.
          \item[Sum of a Point and a Vector] Produces a \emph{point}.
            Add the corresponding coordinates.
          \item[Difference Between Two Points] Produces a \emph{vector}.
            Subtract the corresponding coordinates.
        \end{description}

      \subsection{Multiplying and Dividing a Vector by a Scalar}
        The product of a vector and a scalar is a \emph{vector}.
        Multiply each of the coordinates by the scalar.
        This increases the magnitude of a vector without changing its direction.
        Similarly, we can divide a vector by a scalar.

      \subsection{Magnitude of a Vector}
        The magnitude of a vector is a \emph{scalar}.
        It is the square root of the sum of the squares of its coordinates.
        Written $\lvert v \rvert$.

        If $v$ is the vector $(v_x, v_y, v_z)$, then 
        \begin{align*}
          \lvert v \rvert = \sqrt{v_x^2 + v_y^2 + v_z^2}
        \end{align*}

        \begin{example}
          If $v = (3, 0, 4)$, then
          \begin{align*}
            \lvert v \rvert = \sqrt{v_x^2 + v_y^2 + v_z^2}
                            = \sqrt{3^2 + 0^2 + 4^2}
                            = \sqrt{9 + 0 + 16} = \sqrt{25} = 5
          \end{align*}
        \end{example}

        The magnitude of a vector represents its size, without specifying its direction.

      \subsection{Normalising a Vector}
        A \concept{normalised vector} is a vector with a magnitude of 1.
        To normalise a vector, we divide it by its magnitude.
        \begin{align*}
          \mathbf{n} = \frac{v}{\lvert v \rvert}
        \end{align*}

      \subsection{Dot Product of Two Vectors}
        The \concept{dot product} (also called the \concept{inner product}) of two vectors
        is a \emph{scalar}.
        It is the sum of the products of the two vectors.
        Written $u \cdot v$.

        If $u$ and $v$ are vectors, and $u$ is defined as $(u_x, u_y, u_z)$ and $v$ as $(v_x, v_y, v_z)$,
        then
        \begin{align*}
          u \cdot v = (u_x \cdot v_x) + (u_y \cdot v_y) + (u_z \cdot v_z)
        \end{align*}

        \begin{sidenote}{Angle Between Two Vectors}
          The angle $\theta$ between two vectors can be calculated using:
          \begin{align*}
            \cos(\theta) = \frac{u \cdot v}{\lvert u \rvert \lvert v \rvert}
          \end{align*}
        \end{sidenote}

      \pagebreak

      \subsection{Cross Product of Two Vectors}
        The cross product of two vectors is a \emph{vector}.
        Each coordinate is calculated as the difference between the products of the
        other two vectors.
        Written $u \times v$.

        If $u$ and $v$ are vectors, and $u$ is defined as $(u_x, u_y, u_z)$ and $v$ as $(v_x, v_y, v_z)$,
        then
        \begin{align*}
          u \times v &= \bigl((u_y \cdot v_z - u_z \cdot v_y), - (u_x \cdot v_z - u_z \cdot v_x), (u_x \cdot v_y - u_y \cdot v_x)\bigr)\\
                     &= \bigl((u_y \cdot v_z - u_z \cdot v_y), (u_z \cdot v_x - u_x \cdot v_z), (u_x \cdot v_y - u_y \cdot v_x)\bigr)
        \end{align*}

        Remembering this exact formula is tricky.
        One way to do it is to write them down in columns, and then ignore the column
        for the specific coordinate.
        With that, every even column should be negated.

        \begin{example}
          If $u = (1, 2, 3)$ and $v = (-1, 0, 1)$, then
          \begin{align*}
            u \times v &= \bigl((u_y \cdot v_z - u_z \cdot v_y), - (u_x \cdot v_z - u_z \cdot v_x),
              (u_x \cdot v_y - u_y \cdot v_x)\bigr)\\
                       &= \bigl((2 \cdot 1 - 3 \cdot 0), - (1 \cdot 1 - (3)(-1)), (2 \cdot 1 - 3 \cdot 0)\bigr)\\
                       &= \bigl(2, - (4), 2\bigr)\\
                       &= (2, -4, 2)
          \end{align*}
        \end{example}

        The cross product is useful for calculating a vector normal (at right angles)
        to two vectors.
        You can see that $u \times v = n$ is at right angles to $u$ and $v$ by calculating
        the dot product of $n$~and~$u$ and $n$~and~$v$.
        If $\cos(\theta) = 0$, then $\theta = 90$ degrees.

      \subsection{Lines}
        In 2D, the equation of a line is
        \begin{center}
          $y = mx + c$\\
          $ax + by + c = 0$
        \end{center}

        In 3D, to specify the equation of a line we can use a set of Cartesian equations:
        \begin{align*}
          \frac{x - q_1}{d_1} = \frac{y - q_2}{d_2} = \frac{z - q_3}{d_3}
        \end{align*}
        where $q_1$, $q_2$, $q_3$, $d_1$, $d_2$, and $d_3$ are scalars,
        and $x$, $y$ and $z$ are the coordinates of any point on the line.

        An alternative way of specifying lines in three dimensions is using
        \concept{parametric~equations}.
        Suppose we have a point $q$ on the line, and a vector $d$ pointing in the direction
        of the line.
        Then the following parametric equation specifies the coordinates of any point $p$
        on the line:
        \begin{align*}
          p = q + \gamma d
        \end{align*}
        where $\gamma$ is a scalar value.

        This is related to the Cartesian equations:
        \begin{align*}
          (x, y, z) = (q_1, q_2, q_3) + \gamma (d_1, d_2, d_3)
        \end{align*}
        If we separate the coordinates into different equations, and express those equations
        in terms of $\gamma$, we get the Cartesian equations.

        If one of the components of the direction vector is zero, the Cartesian equations
        turn out to be the equation of a line in 2 dimensions.

        We can also specify a line by means of a parametric equation if we have two points
        $q$ and $r$ on the line:
        \begin{align*}
          p = q + \gamma (r - q)
        \end{align*}
        Here, $r - q$ determines the direction vector.
        For values of $\gamma$ between 0 and 1, we have all the points on the line~segment
        between $r$ and $q$.

        Sometimes, the parametric equation of a line is rearranged in the form
        \begin{align*}
          p = \gamma r + (1 - \gamma)q
        \end{align*}

      \subsection{Planes}
        A \concept{plane} (in three dimensions) can be specified by means of an equation
        of the form
        \begin{align*}
          ax + by + cz + d = 0
        \end{align*}
        where $a$, $b$, $c$, and $d$ are scalars, and $x$, $y$ and $z$ are the coordinates
        of any point in the plane.
        (As it turns out, $a$, $b$, and $c$ are the components of a vector normal -- perpendicular -- to the plane).

        The equation $x + 2y + 3z - 4 = 0$ is the equation of a plane with the vector $(1, 2, 3)$
        perpendicular to it.

        \begin{sidenote}{Determining the Equation of a Plane}
          $ $\vspace{-1em}
          \begin{enumerate}
            \item Given a point $q$ on the plane, and a vector $n$ normal to the plane,
              the equation
              \begin{align*}
                n_1x + n_2y + n_3z + d = 0
              \end{align*}
              will define the plane.
              To get the scalar for $d$, substitute the coordinates of point $q$ and solve.
            \item Given a point $q$ on the plane, and two vectors $u$ and $v$ in the plane,
              the cross product $u \times v$ will give a normal to the plane.
              Then use the method above.
            \item Given three non-collinear points $q$, $r$ and $s$ in the plane,
              use two pairs of these points to determine two vectors in the plane
              by subtracting the points from one another.
              Then use the methods above to find a normal to the plane and then determine
              the equation of the plane.
            \item Given a line in the plane and a point $r$ in the plane.
              The line must be specified in some way: a set of Cartesian equations
              or a parametric equation.
              Then two vectors in the plane will be the direction vector $d = (d_1, d_2, d_3)$
              and the vector $q - r$.
              Now we have two vectors and a point on the plane, and can follow the previous methods
              to determine the equation of the plane.
          \end{enumerate}
        \end{sidenote}

        Planes can also be specified by means of parametric equations:
        \begin{enumerate}
          \item Given a point $q$ and two (linearly independent) vectors $u$ and $v$ in the plane:
            the equation
            \begin{align*}
              p = q + \alpha u + \beta v
            \end{align*}
            will specify all points $p$ on the plane.
          \item Given three non-collinear points $q$, $r$ and $s$, $q - r$ and $q - s$
            will give two linearly independent vectors in the plane.
            The equation
            \begin{align*}
              p = q + \alpha (q - r) + \beta (q - s)
            \end{align*}
            will then specify all points $p$ on the plane.
            For values of $\alpha$ and $\beta$ between 0 and 1, this equation gives all points
            in the triangle formed by $q$, $r$ and $s$.
        \end{enumerate}

      \subsection{Matrices}
        A two-dimensional \concept{matrix} is a table of values with a fixed number of rows
        and columns.
        The values are generally placed in a single pair of extended square brackets.

        \begin{definition}{Transpose of a Matrix}
          The \concept{transpose} of a matrix is another matrix with all its row and column values
          swapped.
          If $A$ is an $m \times n$ matrix, then the transpose of $A$, written $A^T$,
          is the $n \times m$ matrix with the entries swapped.
        \end{definition}

        \begin{definition}{Matrix Addition and Subtraction}
          Matrix addition is only defined for two matrices that have the same number of rows
          and columns, and their sum is another matrix,
          with the same number of rows and columns.
        \end{definition}

        The product of two matrices is used for transformations.
        If we want to translate, scale, or rotate an object, we multiply each of its points
        by the applicable transformation matrix.

        \begin{definition}{Determinant of a Matrix}
          The \concept{determinant} of a matrix is a scalar, and can only be calculated for
          square matrices.
          \begin{align*}
            \lvert A \rvert &= \sum \alpha_{ij} C_{ij}\\
            C_{ij} &= (-1)^{i + j} M_{ij}
          \end{align*}
          where $M_{ij}$ is the minor of matrix $A$ -- the matrix formed by eliminating
          row~$i$ and column~$j$ from $A$.
        \end{definition}

        If the determinant of a matrix is 0, the vectors inside the matrix are linearly dependent.

    \section{Three Dimensional Primitives}
      \begin{sidenote}{Features of 3D Objects That Work with Existing Graphics}
        $ $\vspace{-1em}
        \begin{itemize}
          \item The objects are described by their surfaces and can be thought of as being hollow.
          \item The objects can be specified through a set of vertices in three dimensions.
          \item The objects are either composed of, or can be approximated by, flat, convex polygons.
        \end{itemize}
      \end{sidenote}

      \begin{sidenote}[title after break={Commonly Used 3D Primitives (Continued)}]{Commonly Used 3D Primitives}
        $ $\vspace{-1em}
        \begin{descriptimize}
          \item[Points] The simplest form of a 3D primitive.
            Represent a single position in 3D space, and have no size or shape.
            Typically used as reference coordinates,
            or as building blocks for constructing other geometric objects.
          \item[Lines and Line Segments] Defined by two points in 3D space.
            A line extends infinitely in both directions, while a line segment has a finite length
            between two endpoints.
            Used to represent edges or boundaries of objects.
            \pagebreak
          \item[Polygons] Flat, two-dimensional shapes composed of straight line segments.
            Used to create surfaces and faces of objects.
            Triangles and quadrilaterals are the most commonly used polygon types.
          \item[Cubes and Boxes] Three dimensional objects with six square faces,
            all of equal size.
            Often used as basic building blocks for constructing more complex objects.
          \item[Spheres] Perfectly round objects with all points on the surface equidistant
            from a central point.
            Defined by their centre point and radius.
          \item[Cylinders] Tube-like objects with circular bases at both ends and a curved surface
            connecting them.
            Defined by their base centre points, radii, and height.
          \item[Cones] Similar to cylinders, but have a single circular base and a curved surface
            that tapers to a point called the apex.
            Defined by their base centre point, radius, and height.
        \end{descriptimize}
      \end{sidenote}

    \section{Frames in WebGL}
      \concept{Coordinates} are used to specify the position of a point in a given space.
      They define a reference system or coordinate system that provides a framework
      for measuring and locating points.

      \begin{definition}{Frame}
        Also known as a \concept{reference frame} or \concept{coordinate frame}.

        An abstract construct used to define the orientation and position of objects
        within a space.
        It consists of a coordinate system, along with an origin point and a set of basis vectors.
        The basis vectors, typically orthogonal to each other, define the directions
        of the coordinate axes.
      \end{definition}

      In 2D graphics, a frame is defined by an origin point and two basis vectors
      (often referred to as the $u$-axis and $v$-axis).
      In 3D graphics, a frame is defined by an origin point and three basis vectors
      (often referred to as the $x$-axis, $y$-axis, and $z$-axis).
      These basis vectors determine the orientation and alignment of the coordinate axes
      within the space.

      Frames allow us to describe the position, orientation, and transformations of objects
      relative to a specific reference system.
      By applying transformations to the basis vectors and origin point of a frame,
      we can change its position, rotation, and scale, thereby affecting the objects
      attached to that frame.

      The application programmer usually works with two frames: the \concept{model frame}
      and the \concept{view frame}.
      Because there is an affine transformation that corresponds to each change of frame,
      there are $4 \times 4$ matrices that represent the transformation from model coordinates
      to world coordinates, and from world coordinates to eye coordinates.
      These transformations are usually concatenated together into the
      \concept{model-view transformation}, which is specified by the model-view matrix.

      \pagebreak

      \begin{sidenote}{Order of Frames in the Pipeline}
        Note that the names have changed between the versions, for the first two.

        \begin{descriptenum}
          \item[Model Coordinates] (In the older version, \concept{object (or model) coordinates}).
            In most applications, we tend to specify or use
            an object with a convenient size, orientation, and location in its own frame,
            called the \concept{model~frame}.
          \item[Object (or World) Coordinates] (In the older version,
            \concept{world (or application) coordinates}).
            A scene may comprise many objects.
            The application program generally applies a sequence of transformations to each object
            to size, orient, and position it within a frame that is appropriate for that particular
            application.
            The application frame is called the \concept{world frame},
            and the values are in \concept{world coordinates}.
            If we do not model with predefined objects, or apply any transformations before
            we specify our geometry, model and world/object coordinates are the same.
          \item[Eye (or Camera) Coordinates] Virtually all graphics systems use a frame
            whose origin is the center of the camera's lens, and whose axes are aligned
            with the sides of the camera.
            This frame is called the \concept{camera (or eye) frame}.
            There are $4 \times 4$ matrices that represent the transformation from the
            model coordinates to world coordinates, and from world coordinates to eye coordinates.
            These transformations are normally concatenated together into the
            \concept{model-view transformation}, which is specified by the model-view matrix.
          \item[Clip Coordinates] Once objects are in eye coordinates, WebGL must check whether
            they lie within the view volume.
            If an object does not, it is clipped from the scene prior to rasterisation.
            WebGL can carry out this process most efficiently if it fist carries out a projection
            transformation that brings all potentially visible objects into a cube centered
            at the origin in \concept{clip coordinates}.
          \item[Normalised Device Coordinates] At this stage, vertices are still represented
            in homogeneous coordinates.
            The division by the $w$ component, called \concept{perspective~division},
            yields 3D representations in \concept{normalised device coordinates}.
          \item[Window (or Screen) Coordinates] Takes a position in normalised device coordinates,
            and, taking into account the viewport, creates a 3D representation in
            \concept{window coordinates}.
            Window coordinates are measured in units of pixels on the display,
            but retain depth information.
            If we remove the depth coordinate, we are working with 2D \concept{screen coordinates}.
        \end{descriptenum}
      \end{sidenote}

    \section{Modelling a Coloured Cube}
      \begin{sidenote}{Order of Vertices}
        We have to be careful about the order in which we specify our vertices
        when we are defining a three-dimensional polygon.

        We call a face \concept{outward~facing} if the vertices are traversed in a counterclockwise
        order when the face is viewed from the outside.
        This method is also known as the \concept{right-hand~rule}, because if you orient
        the fingers of your right hand in the direction the vertices are traversed,
        the thumb points outward.
      \end{sidenote}

    \section{Affine Transformations}
      \begin{definition}{Transformation}
        A \concept{transformation} is a function that takes a point (or vector)
        and maps it into another point (or vector).
      \end{definition}

      \begin{definition}{Affine Space}
        An extension of the vector pace that includes an additional type of object: the point.
      \end{definition}

      If we use homogeneous coordinate representations, we can represent both vectors and points
      as four-dimensional column matrices, and we can define the transformation with a single
      function:
      \begin{align*}
        \mathbf{q} &= f(\mathbf{p})\\
        \mathbf{v} &= f(\mathbf{u})
      \end{align*}

      When we work with homogeneous coordinates, any affine transformation can be represented
      by a $4 \times 4$~matrix that can be applied to a point or vector by pre-multiplication.
      \begin{align*}
        \mathbf{q} &= T\mathbf{p}
      \end{align*}

      All affine transformations preserve lines.

    \section{Translation, Rotation, and Scaling}
      \begin{definition}{Translation}
        An operation that displaces points by a fixed distance in a given direction.

        To specify a translation, we need only specify a displacement vector $d$,
        because the transformed points are given by
        \begin{align*}
          p' = p + d
        \end{align*}
        for all points $p$ on the object.
      \end{definition}

      \begin{definition}{Rotation}
        An operation that rotates points by a fixed angle about a point or line.

        The positive direction of rotation is counter-clockwise.
      \end{definition}

      Rotation and translation are known as \concept{rigid-body~transformations}.
      No combination of rotations and translations can alter the shape or volume of an object:
      they can alter only the object's location and orientation.

      \pagebreak

      \begin{definition}{Scaling}
        An affine non-rigid-body transformation by which we can make an object bigger or smaller.

        Scaling transformations have a \concept{fixed point}: a point that is unaffected by the
        transformation.
        To specify a scaling, we can specify the fixed point, a direction in which we wish to scale,
        and a scale factor ($\alpha$).
        If $\alpha > 1$, the object gets longer in the specified direction.
        If $0 \leq \alpha < 1$, the object gets shorter in that direction.
        Negative values of $\alpha$ give us \concept{reflection} about the fixed point
        in the scaling direction.
        \begin{descriptimize}
          \item[Uniform Scaling] The scaling factor in all directions is identical.
            The shape of the scaled object is preserved.
          \item[Non-uniform Scaling] The scaling factor or each direction need not be identical.
            The shape of the scaled object is distorted.
        \end{descriptimize}
      \end{definition}

      \begin{definition}{Shearing}
        A transformation that slants or skews an object along one or more axes.
        It involves modifying the coordinates of points by a certain proportion
        in relation to their position along other axes.
        Shearing can by applied independently along the $x$-axis, $y$-axis, or both.
      \end{definition}

      \begin{definition}{Reflection}
        Involves creating a mirror image of an object across a line or plane called the mirror axis.
        It can be performed along any line or plane, including the $x$-axis, $y$-axis,
        or an arbitrary line or plane defined by equations.
        Reflection changes the orientation of the object while preserving its shape.
      \end{definition}

    \section{Transformations in Homogeneous Coordinates}
      Within a frame, each affine transformation is represented by a $4 \times 4$ matrix of the form
      \begin{align*}
        A = \begin{bmatrix}
          \alpha_{11} & \alpha_{12} & \alpha_{13} & \alpha_{14} \\
          \alpha_{21} & \alpha_{22} & \alpha_{23} & \alpha_{24} \\
          \alpha_{31} & \alpha_{32} & \alpha_{33} & \alpha_{34} \\
          0 & 0 & 0 & 1
        \end{bmatrix}
      \end{align*}
      The final row of the matrix does not depend on a particular transformation,
      but rather forces the fourth component of the transformation to retain the value 1.

      \pagebreak

      \subsection{Translation}
        Translation displaces points to new positions defined by a displacement vector.

        \begin{theorem}{Translation Matrix}
          $\mathbf{p}' = \mathbf{Tp}$
          \begin{align*}
            \mathbf{T}(\alpha_x, \alpha_y, \alpha_z) = \begin{bmatrix}
              1 & 0 & 0 & \alpha_x \\
              0 & 1 & 0 & \alpha_y \\
              0 & 0 & 1 & \alpha_z \\
              0 & 0 & 0 & 1
            \end{bmatrix}
          \end{align*}
        \end{theorem}

        \begin{theorem}{Inverse Translation Matrix}
          \begin{align*}
            \mathbf{T}^{-1}(\alpha_x, \alpha_y, \alpha_z) = \mathbf{T}(-\alpha_x, -\alpha_y, -\alpha_z)
              = \begin{bmatrix}
                1 & 0 & 0 & -\alpha_x \\
                0 & 1 & 0 & -\alpha_y \\
                0 & 0 & 1 & -\alpha_z \\
                0 & 0 & 0 & 1
              \end{bmatrix}
          \end{align*}
        \end{theorem}

      \subsection{Scaling}
        For both scaling and rotation, there is a fixed point that is unchanged by the
        transformation.
        A scaling matrix with a fixed point at the origin allows for independent scaling
        along the coordinate axes.

        \begin{theorem}{Scaling Matrix}
          $\mathbf{p}' = \mathbf{Sp}$.

          If $\beta$ is the scaling factor for the different coordinates, then:
          \begin{align*}
            \mathbf{S} = \mathbf{S}(\beta_x, \beta_y, \beta_z)
            = \begin{bmatrix}
              \beta_x & 0 & 0 & 0 \\
              0 & \beta_y & 0 & 0 \\
              0 & 0 & \beta_z & 0 \\
              0 & 0 & 0 & 1
            \end{bmatrix}
          \end{align*}
        \end{theorem}

        \begin{theorem}{Inverse Scaling Matrix}
          We obtain the inverse of a scaling matrix by applying the reciprocals of the scale factors:
          \begin{align*}
            \mathbf{S}^{-1}(\beta_x, \beta_y, \beta_z)
            = \mathbf{S}\left(\frac{1}{\beta_x}, \frac{1}{\beta_y}, \frac{1}{\beta_z}\right)
          \end{align*}
        \end{theorem}

      \subsection{Rotation}
        Matrix multiplication is not commutative.
        Rotation about the $x$-axis by an angle~$\theta$ followed by rotation about the $y$-axis
        by an angle~$\theta$ does not lead to the same result as we would obtain if we reverse
        the operations.

        \begin{theorem}{Rotation Matrices}
          \begin{align*}
            \mathbf{R}_z = \mathbf{R}_z(\theta) &= \begin{bmatrix}
              \cos\theta & -\sin\theta & 0 & 0 \\
              \sin\theta & \cos\theta & 0 & 0 \\
              0 & 0 & 1 & 0 \\
              0 & 0 & 0 & 1
            \end{bmatrix}\\
            \mathbf{R}_x = \mathbf{R}_x(\theta) &= \begin{bmatrix}
              1 & 0 & 0 & 0 \\
              0 & \cos\theta & -\sin\theta & 0 \\
              0 & \sin\theta & \cos\theta & 0 \\
              0 & 0 & 0 & 1
            \end{bmatrix}\\
            \mathbf{R}_y = \mathbf{R}_y(\theta) &= \begin{bmatrix}
              \cos\theta & 0 & \sin\theta & 0\\
              0 & 1 & 0 & 0\\
              -\sin\theta & 0 & \cos\theta & 0\\
              0 & 0 & 0 & 1
            \end{bmatrix}
          \end{align*}
        \end{theorem}

        \begin{theorem}{Inverse Rotation Matrix}
          A rotation by $\theta$ can always be undone by a subsequent rotation by $-\theta$.
          Which means,
          \begin{align*}
            \mathbf{R}^{-1}(\theta) = \mathbf{R}(-\theta)
          \end{align*}

          Because all cosine terms are on the diagonal, and all sine terms are off it,
          we can use the trig identities
          \begin{align*}
            \cos(-\theta) &= \cos\theta\\
            \sin(-\theta) &= -\sin\theta
          \end{align*}
          to find
          \begin{align*}
            \mathbf{R}^{-1}(\theta) = \mathbf{R}^{T}(\theta)
          \end{align*}
        \end{theorem}

        \begin{sidenote}{Product of Rotations}
          We can construct any desired rotation matrix, with a fixed point at the origin,
          as a product of individual rotations about the three axes:
          \begin{align*}
            \mathbf{R} = \mathbf{R}_z\mathbf{R}_y\mathbf{R}_x
          \end{align*}

          NOTE: Order matters, it's z, y, x
        \end{sidenote}

        Using the fact that the transpose of a product is the product of the transposes
        in the reverse order, for any rotation matrix
        \begin{align*}
          \mathbf{R}^{-1} = \mathbf{R}^T
        \end{align*}

        A matrix whose inverse is equal to its transpose is called an \concept{orthogonal matrix}.
        Normalised orthogonal matrices correspond to rotations about the origin.

        \begin{example}[Some Transformation Matrices]
          \begin{minipage}[t]{0.19\linewidth}
            \begin{center}
              \captionsetup{type=figure, justification=centering}
              \begin{align*}
                \begin{bmatrix}
                  1 & 0 & 0 & 0 \\
                  0 & 1 & 0 & 0 \\
                  0 & 0 & 1 & 0 \\
                  0 & 0 & 0 & 3
                \end{bmatrix}
              \end{align*}
              \vspace{-1em}
              \caption*{Uniform Scaling}
            \end{center}
          \end{minipage}
          \begin{minipage}[t]{0.19\linewidth}
            \begin{center}
              \captionsetup{type=figure, justification=centering}
              \begin{align*}
                \begin{bmatrix}
                  3 & 0 & 0 & 0 \\
                  0 & 3 & 0 & 0 \\
                  0 & 0 & 1 & 0 \\
                  0 & 0 & 0 & 3
                \end{bmatrix}
              \end{align*}
              \vspace{-1em}
              \caption*{Non-Uniform Scaling}
            \end{center}
          \end{minipage}
          \begin{minipage}[t]{0.19\linewidth}
            \begin{center}
              \captionsetup{type=figure, justification=centering}
              \begin{align*}
                \begin{bmatrix}
                  0 & -1 &  0 & 0 \\
                  0 &  0 & -1 & 0 \\
                  -1 &  0 &  0 & 0 \\
                  0 &  0 &  0 & 1
                \end{bmatrix}
              \end{align*}
              \vspace{-1em}
              \caption*{Reflection}
            \end{center}
          \end{minipage}
          \begin{minipage}[t]{0.19\linewidth}
            \begin{center}
              \captionsetup{type=figure, justification=centering}
              \begin{align*}
                \begin{bmatrix}
                  \sqrt{2} & -\sqrt{2} & 0 & 0 \\
                  \sqrt{2} & \sqrt{2} & 0 & 0 \\
                  0 & 0 & 2 & 0 \\
                  0 & 0 & 0 & 2
                \end{bmatrix}
              \end{align*}
              \vspace{-1em}
              \caption*{Rotation\\About~$z$}
            \end{center}
          \end{minipage}
          \begin{minipage}[t]{0.19\linewidth}
            \begin{center}
              \captionsetup{type=figure, justification=centering}
              \begin{align*}
                \begin{bmatrix}
                  0 & -1 &  0 & 0 \\
                  0 &  0 & -1 & 0 \\
                  1 &  0 &  0 & 0 \\
                  0 &  0 &  0 & 1
                \end{bmatrix}
              \end{align*}
              \vspace{-1em}
              \caption*{Rotation}
            \end{center}
          \end{minipage}
        \end{example}

    \section{Concatenation of Transformations}
      We can combine affine transformations by multiplying together, or \concept{concatenating},
      sequences of the basic transformations.

      Note that the order of the steps means which is applied first.
      If we want to apply $\mathbf{A}$, then $\mathbf{B}$, then $\mathbf{C}$,
      that would mean:
      \begin{enumerate}[nosep]
        \item $p$
        \item $\mathbf{A}(p)$
        \item $\mathbf{B}(\mathbf{A}(p))$
        \item $\mathbf{C}(\mathbf{B}(\mathbf{A}(p)))$
        \item $\mathbf{CBA}p$
      \end{enumerate}

      \subsection{Rotation About a Fixed Point}
        \begin{example}
          Consider a cube with its centre at $\mathbf{p}_f$ and its sides aligned with the axes.
          We want to rotate the cube without changing $x$ and $y$ values,
          but this time about its centre $\mathbf{p}_f$, which becomes the fixed point
          of the transformation.
          If $\mathbf{p}_f$ were the origin, we would know how to solve the problem:
          simply use $\mathbf{R}_z(\theta)$.

          So:
          \begin{enumerate}[nosep]
            \item Move the cube to the origin: $\mathbf{T}(-\mathbf{p}_f)$.
            \item Apply $\mathbf{R}_z(\theta)$.
            \item Move the object back, so that its centre is again at $\mathbf{p}_f$:
              $\mathbf{T}(\mathbf{p}_f)$.
          \end{enumerate}
          \begin{align*}
            \mathbf{M} = \mathbf{T}(\mathbf{p}_f)\mathbf{R}_z(\theta)\mathbf{T}(-\mathbf{p}_f)
          \end{align*}
        \end{example}

      \subsection{The Instance Transformation}
        Instead of defining all the objects separately, with their own attributes,
        we can specify each of the object types once at a convenient size, in a convenient
        place, and with a convenient orientation.
        Each occurrence of an object in the scene is an \concept{instance} of that object's
        prototype, and we can obtain the desired size, orientation, and location
        by applying an affine transformation -- the \concept{instance~transformation} --
        to the prototype.

        Objects are usually defined in their own frames, with the origin at the centre of mass,
        and the sides aligned with the model frame axis.

        First, we scale the object to the desired size, then we orient it, and then translate it.
        \begin{align*}
          \mathbf{M} = \mathbf{TRS}
        \end{align*}

      \subsection{Rotation About an Arbitrary Axis}
        To rotate an object by an angle~$\theta$ about an arbitrary axis, we carry out at most
        two rotations to align the axis of rotation with the axis -- say, the $z$-axis --
        then rotate $\theta$ about that axis, and then undo the two rotations that
        did the aligning.

        \begin{enumerate}[nosep]
          \item Rotate about $x$-axis: $\mathbf{R}_x$.
          \item Rotate about $y$-axis: $\mathbf{R}_y$.
          \item Do the required rotation: $\mathbf{R}_z(\theta)$.
          \item Undo $y$ rotation: $\mathbf{R}^{-1}_y$.
          \item Undo $x$ rotation: $\mathbf{R}^{-1}_x$.
        \end{enumerate}
        \begin{align*}
          \mathbf{R} = \mathbf{R}^{-1}_x \mathbf{R}^{-1}_y \mathbf{R}_z(\theta) \mathbf{R}_y \mathbf{R}_x
        \end{align*}

    \section{Transformation Matrices in WebGL}
      In a modern implementation of OpenGL, the application programmer can choose not only
      which frames to use
      (model-, world-, and eye-frame),
      but also where to carry out the transformations between frames
      (application or vertex shader).
      Although very few state variables are predefined in WebGL, once we specify various
      attributes and matrices, they effectively define the state of the system.
      Thus, when a vertex is processed, how it is processed is determined by the values
      of these state variables.

      The two transformations we will use the most often are the
      \begin{descriptimize}
        \item[Model-View Transformation] Brings representations of geometric objects
          from the application or object frame to the camera frame.
        \item[Projection Transformation] Both carries out the desired projection,
          and change the representation to clip coordinates.
      \end{descriptimize}

      \begin{definition}{Uniform Qualified Variables}
        Parameters that remain the same for all vertices in a primitive.
      \end{definition}

  \rulechapterend

\end{document}
