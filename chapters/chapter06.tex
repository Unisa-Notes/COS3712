\providecommand{\main}{..}
\documentclass[../COS3712_Notes.tex]{subfiles}

\begin{document}
  \setcounter{chapter}{5}
  \chapter{Lighting and Shading}
    \concept{Local lighting modules}, as opposed to \concept{\emph{global} lighting modules},
    allow us to compute the shade to assign to a point on a surface,
    independent of any other surfaces in the scene.
    The calculations depend only on the material properties assigned to the surface,
    the local geometry of the surface, and the locations and properties of the light sources.

    We have choices as to where to apply a given lighting model:
    in the application, in the vertex shader, or in the fragment shader.

    \section{Light and Matter}
      From a physical perspective, a surface can either emit light by self-emission,
      as a light-bulb does,
      or reflect light from other surfaces that illuminate it.
      When we look at a point on an object, the colour we see is determined by multiple
      interactions among light sources and reflective surfaces.
      These interactions can be viewed as a recursive process.
      The algorithm for calculating this (the \concept{rendering equation}) cannot be solved
      analytically in the general case.
      Instead, we focus on a simpler rendering model, based on the \concept{Phong reflection model},
      that provides a compromise between physical correctness and efficient calculation.

      Rather than looking at a global energy balance, we follow rays of light
      from light-emitting (or self-luminous) surfaces that we call \concept{light~sources}.
      We then model what happens to these rays as they interact with reflecting surfaces
      in the scene.

      \begin{sidenote}{Groups of Interactions Between Light and Materials}
        $ $\vspace{-1em}
        \begin{descriptimize}[nosep]
          \item[Specular~surfaces] appear shiny because most of the light
            that is reflected or \concept{scattered} is in a narrow range of angles
            close to the angle of reflection.
            Mirrors are \concept{perfectly specular surfaces}: the light from an incoming
            light ray may be partially absorbed, but all reflected light from a given angle
            emerges at a single angle.
          \item[Diffuse~surfaces] are characterised by reflected light being scattered
            in all directions.
            Walls painted with matte or flat paint would be an example.
            \concept{Perfectly diffuse surfaces} scatter light equally in all directions,
            and thus a flat, perfectly diffuse surface appears the same to all viewers.
          \item[Translucent surfaces] allow some light to penetrate the surface
            and to emerge from another location on the object.
            This process of \concept{refraction} characterises glass and water.
        \end{descriptimize}
      \end{sidenote}

\end{document}
