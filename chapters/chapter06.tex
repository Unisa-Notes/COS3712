\providecommand{\main}{..}
\documentclass[../COS3712_Notes.tex]{subfiles}

\begin{document}
  \setcounter{chapter}{5}
  \chapter{Lighting and Shading}
    \concept{Local lighting modules}, as opposed to \concept{\emph{global} lighting modules},
    allow us to compute the shade to assign to a point on a surface,
    independent of any other surfaces in the scene.
    The calculations depend only on the material properties assigned to the surface,
    the local geometry of the surface, and the locations and properties of the light sources.

    We have choices as to where to apply a given lighting model:
    in the application, in the vertex shader, or in the fragment shader.

    \section{Light and Matter}
      From a physical perspective, a surface can either emit light by self-emission,
      as a light-bulb does,
      or reflect light from other surfaces that illuminate it.
      When we look at a point on an object, the colour we see is determined by multiple
      interactions among light sources and reflective surfaces.
      These interactions can be viewed as a recursive process.
      The algorithm for calculating this (the \concept{rendering equation}) cannot be solved
      analytically in the general case.
      Instead, we focus on a simpler rendering model, based on the \concept{Phong reflection model},
      that provides a compromise between physical correctness and efficient calculation.

      Rather than looking at a global energy balance, we follow rays of light
      from light-emitting (or self-luminous) surfaces that we call \concept{light~sources}.
      We then model what happens to these rays as they interact with reflecting surfaces
      in the scene.

      \begin{sidenote}{Groups of Interactions Between Light and Materials}
        $ $\vspace{-1em}
        \begin{descriptimize}[nosep]
          \item[Specular~surfaces] appear shiny because most of the light
            that is reflected or \concept{scattered} is in a narrow range of angles
            close to the angle of reflection.
            Mirrors are \concept{perfectly specular surfaces}: the light from an incoming
            light ray may be partially absorbed, but all reflected light from a given angle
            emerges at a single angle.
          \item[Diffuse~surfaces] are characterised by reflected light being scattered
            in all directions.
            Walls painted with matte or flat paint would be an example.
            \concept{Perfectly diffuse surfaces} scatter light equally in all directions,
            and thus a flat, perfectly diffuse surface appears the same to all viewers.
          \item[Translucent surfaces] allow some light to penetrate the surface
            and to emerge from another location on the object.
            This process of \concept{refraction} characterises glass and water.
        \end{descriptimize}
      \end{sidenote}

    \section{Light Sources}
      Light can leave a surface through two fundamental processes: self-emission and reflection.
      A general light source can be characterised by a six-variable \concept{illumination function}
      $I(x, y, z, \theta, \phi, \lambda)$.
      We need two angles to specify a direction, and we are assuming that each frequency
      can be considered independently.

      \subsection{Colour Sources}
        Light sources emit different amounts of light at different frequencies,
        and their directional properties can vary with frequency as well.
        For most applications, we can model light sources as having three components
        -- red, green, and blue --
        and we can use each of the three colour sources to obtain the corresponding
        colour components that a human observer sees.
        We describe a source through a three-component intensity or \concept{luminance} function:
        \begin{align*}
          \mathbf{I} = \begin{bmatrix}
            I_r \\
            I_g \\
            I_b
          \end{bmatrix}
        \end{align*}
        each of whose components is the intensity of the independent red, green, and blue components.

        There are four basic types of sources:
        \begin{descriptenum}
          \item[Ambient Light] Lights that are designed and positioned to provide uniform
            illumination throughout the room.
            Ambient illumination is characterised by an intensity, $\mathbf{I}_a$,
            that is identical at every point in the scene.
            Although every point in our scene receives the same illumination from $\mathbf{I}_a$,
            each surface can reflect this light differently.
          \item[Point Sources] An ideal \concept{point source} emits light equally
            in all directions.
            The intensity of illumination received from a point source is proportional to the
            inverse square root of the distance between the source and surface.
          \item[Spotlights] Characterised by a narrow range of angles through which light is
            emitted.
            We can construct a simple spotlight from a point source by limiting the angles
            at which light from the source can be seen.
            More realistic spotlights are characterised by the distribution of light
            within the cone, usually with most of the light concentrated in the
            centre of the cone.
          \item[Distant Light] If the light source is far from the surface, the vector
            does not change much as we move from point to point.
            All rays are parallel, and we replace the location (point) of the light source
            with the direction (vector) of the light.
        \end{descriptenum}

\end{document}
