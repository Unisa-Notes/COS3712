\providecommand{\main}{..}
\documentclass[../notes.tex]{subfiles}

\begin{document}
  \setcounter{chapter}{1}
  \chapter{Graphics Programming}
    Two-dimensional graphics can be viewed as a special case of three-dimensional graphics.

    \section{The Sierpinski Gasket}
      \begin{definition}{Sierpinski Gasket}
        The \concept{Sierpinski Gasket} is an object that can be defined recursively and randomly,
        but has properties that are not random.

        Suppose we start with three points in space.
        As long as the points are not collinear, they are the vertices of a unique triangle
        and also define a unique plane.
        Assume this plane is the plane $z = 0$ and that these points are $(x_1, y_1, 0)$,
        $(x_2, y_2, 0)$ and $(x_3, y_3, 0)$.
        The construction proceeds as:
        \begin{enumerate}
          \item Pick an initial point $\mathbf{p} = (x, y, 0)$ at random inside the triangle.
          \item \label{sierp:2} Select one of the three vertices at random.
          \item Find the point $\mathbf{q}$ halfway between $\mathbf{p}$ and the selected vertex.
          \item Display $\mathbf{q}$ by putting some sort of marker,
            such as a small circle,
            at the corresponding location on the display.
          \item Replace $\mathbf{p}$ with $\mathbf{q}$.
          \item Return to step \ref{sierp:2}.
        \end{enumerate}
      \end{definition}

      \begin{definition}{Immediate Mode Graphics}
        As vertices are generated, they are sent directly to the graphics processor
        for rendering on the display.

        Until recently, was the standard method for displaying graphics.
        A consequence is that there is no memory of the geometric data,
        so if we want to redisplay the scene, we have to go through the entire creation
        and display process again.
      \end{definition}

      \begin{definition}{Retained Mode Graphics}
        Compute all the points first and store them in a data structure.
        Then display all the points through a single function call.

        This approach avoids the overhead of sending small amounts of data to the graphics
        processor for each point we generate,
        at the cost of having to store all the data.
        Because the data are stored, we can redisplay the scene, by resending the stored data
        without having to regenerate it.

        Current GPUs allow us to store the generated data directly on the GPU,
        which avoids the bottleneck caused by transferring the data from the CPU to the GPU
        each time we wish to redisplay the scene.
      \end{definition}

\end{document}
