\providecommand{\main}{..}
\documentclass[../COS3712_Notes.tex]{subfiles}

\begin{document}
  \setcounter{chapter}{5}
  \chapter{Lighting and Shading}
    \concept{Local lighting modules}, as opposed to \concept{\emph{global} lighting modules},
    allow us to compute the shade to assign to a point on a surface,
    independent of any other surfaces in the scene.
    The calculations depend only on the material properties assigned to the surface,
    the local geometry of the surface, and the locations and properties of the light sources.

    We have choices as to where to apply a given lighting model:
    in the application, in the vertex shader, or in the fragment shader.

    \section{Light and Matter}
      From a physical perspective, a surface can either emit light by self-emission,
      as a light-bulb does,
      or reflect light from other surfaces that illuminate it.
      When we look at a point on an object, the colour we see is determined by multiple
      interactions among light sources and reflective surfaces.
      These interactions can be viewed as a recursive process.
      The algorithm for calculating this (the \concept{rendering equation}) cannot be solved
      analytically in the general case.
      Instead, we focus on a simpler rendering model, based on the \concept{Phong reflection model},
      that provides a compromise between physical correctness and efficient calculation.

      Rather than looking at a global energy balance, we follow rays of light
      from light-emitting (or self-luminous) surfaces that we call \concept{light~sources}.
      We then model what happens to these rays as they interact with reflecting surfaces
      in the scene.

      \begin{sidenote}{Groups of Interactions Between Light and Materials}
        $ $\vspace{-1em}
        \begin{descriptimize}[nosep]
          \item[Specular~surfaces] appear shiny because most of the light
            that is reflected or \concept{scattered} is in a narrow range of angles
            close to the angle of reflection.
            Mirrors are \concept{perfectly specular surfaces}: the light from an incoming
            light ray may be partially absorbed, but all reflected light from a given angle
            emerges at a single angle.
          \item[Diffuse~surfaces] are characterised by reflected light being scattered
            in all directions.
            Walls painted with matte or flat paint would be an example.
            \concept{Perfectly diffuse surfaces} scatter light equally in all directions,
            and thus a flat, perfectly diffuse surface appears the same to all viewers.
          \item[Translucent surfaces] allow some light to penetrate the surface
            and to emerge from another location on the object.
            This process of \concept{refraction} characterises glass and water.
        \end{descriptimize}
      \end{sidenote}

    \section{Light Sources}
      Light can leave a surface through two fundamental processes: self-emission and reflection.
      A general light source can be characterised by a six-variable \concept{illumination function}
      $I(x, y, z, \theta, \phi, \lambda)$.
      We need two angles to specify a direction, and we are assuming that each frequency
      can be considered independently.

      \subsection{Colour Sources}
        Light sources emit different amounts of light at different frequencies,
        and their directional properties can vary with frequency as well.
        For most applications, we can model light sources as having three components
        -- red, green, and blue --
        and we can use each of the three colour sources to obtain the corresponding
        colour components that a human observer sees.
        We describe a source through a three-component intensity or \concept{luminance} function:
        \begin{align*}
          \mathbf{I} = \begin{bmatrix}
            I_r \\
            I_g \\
            I_b
          \end{bmatrix}
        \end{align*}
        each of whose components is the intensity of the independent red, green, and blue components.

        There are four basic types of sources:
        \begin{descriptenum}
          \item[Ambient Light] Lights that are designed and positioned to provide uniform
            illumination throughout the room.
            Ambient illumination is characterised by an intensity, $\mathbf{I}_a$,
            that is identical at every point in the scene.
            Although every point in our scene receives the same illumination from $\mathbf{I}_a$,
            each surface can reflect this light differently.
          \item[Point Sources] An ideal \concept{point source} emits light equally
            in all directions.
            The intensity of illumination received from a point source is proportional to the
            inverse square root of the distance between the source and surface.
          \item[Spotlights] Characterised by a narrow range of angles through which light is
            emitted.
            We can construct a simple spotlight from a point source by limiting the angles
            at which light from the source can be seen.
            More realistic spotlights are characterised by the distribution of light
            within the cone, usually with most of the light concentrated in the
            centre of the cone.
          \item[Distant Light] If the light source is far from the surface, the vector
            does not change much as we move from point to point.
            All rays are parallel, and we replace the location (point) of the light source
            with the direction (vector) of the light.
        \end{descriptenum}

    \section{The Phong Reflection Model}
      The Phong model uses four vectors to calculate a colour for an arbitrary point $\mathbf{p}$
      on a surface.
      \begin{descriptimize}[nosep]
        \item[$\mathbf{n}$] The normal at $\mathbf{p}$.
        \item[$\mathbf{v}$] In the direction from $\mathbf{p}$ to the viewer or COP.
        \item[$\mathbf{l}$] In the direction of a line from $\mathbf{p}$ to an arbitrary point
          on the source for a distributed light source, or to the point light source.
        \item[$\mathbf{r}$] The direction that a perfectly reflected ray from $\mathbf{l}$ would
          take.
          Determined by $\mathbf{n}$ and $\mathbf{l}$.
      \end{descriptimize}
      If the surface is curved, all four vectors can change as we move from point to point.

      The Phong model supports three types of material-light interactions: ambient, diffuse,
      and specular.
      For each light source, we can have separate ambient, diffuse, and specular components
      for each of the three primary colours.
      Thus, we need nine coefficients to characterise these terms at any point $\mathbf{p}$.
      We can place these coefficients in a $3 \times 3$ illumination matrix for the $i$th
      light source:
      \begin{align*}
        \mathbf{L}_i = \begin{bmatrix}
          L_{i\mathrm{ra}} & L_{i\mathrm{ga}} & L_{i\mathrm{ba}} \\
          L_{i\mathrm{rd}} & L_{i\mathrm{gd}} & L_{i\mathrm{bd}} \\
          L_{i\mathrm{rs}} & L_{i\mathrm{gs}} & L_{i\mathrm{bs}}
        \end{bmatrix}
      \end{align*}

      \subsection{Ambient Reflection}
        The intensity of ambient light $I_a$ is the same at every point on the surface.
        Some of this light is absorbed, and some is reflected.
        The amount reflected is given by the ambient reflection coefficient $R_a = k_a$.
        Because only a positive fraction of the light is reflected,
        we must have
        \begin{align*}
          0 \leq k_a \leq 1
        \end{align*}
        and thus
        \begin{align*}
          I_a = k_a L_a
        \end{align*}
        Here, $L_a$ can be any of the individual light sources,
        or it can be a global ambient term.

      \subsection{Diffuse Reflection}
        A perfectly diffuse reflector scatters the light that it reflects equally in all directions.
        Hence, such a surface appears the same to all viewers.

        The amount of light reflected depends both on the material
        -- because some of the incoming light is absorbed --
        and on the position of the light source relative to the surface.

        Diffuse reflections are characterised by rough surfaces.
        Perfectly diffuse surfaces, which are sometimes called \concept{Lambertian~surfaces},
        are so rough that there is no preferred angle of reflection,
        and can be modelled mathematically with \concept{Lambert's Law}.

        \begin{definition}{Lambert's Law}
          We see only the vertical component of the incoming light.

          The reflected light is proportional to the cosine of the angle between the normal
          and the direction of the light source.
          \begin{align*}
            R_d \propto \cos\theta
          \end{align*}
          where $\theta$ is the angle between the normal at the point of interest $\mathbf{n}$
          and the direction of the light source $\mathbf{l}$.

          If both $\mathbf{l}$ and $\mathbf{n}$ are unit vectors,
          then
          \begin{align*}
            \cos\theta = \mathbf{l} \cdot \mathbf{n}
          \end{align*}
          If we add in a reflection coefficient $k_d$, representing the fraction of incoming
          diffuse light that is reflected,
          we have the diffuse reflection term:
          \begin{align*}
            I_d = k_d (\mathbf{l} \cdot \mathbf{n}) L_d
          \end{align*}
        \end{definition}

      \subsection{Specular Reflection}
        If we employ only ambient and diffuse reflections, our images will be shaded and will
        appear 3D, but all surfaces will look dull.
        Specular reflection adds a highlight that we see reflected from shiny objects.

        Whereas a diffuse surface is rough, a specular surface is smooth.
        As the surface gets smoother, the reflected light is concentrated in a smaller range
        of angles centred about the angle of a perfect reflector.
        The pattern by which light is reflected is not symmetric,
        but depends on the wavelength of the incident light,
        and it changes with the reflection angle.

        The amount of light that the viewer sees depends on the angle $\phi$ between
        $\mathbf{r}$, the direction of a perfect reflector,
        and $\mathbf{v}$, the direction of the viewer.

        The Phong model uses the equation
        \begin{align*}
          I_s = k_s L_s \cos^{\alpha} \phi.
        \end{align*}
        The coefficient $k_s$ is the fraction of the incoming specular light that is reflected.
        The exponent $\alpha$ is a \concept{shininess} coefficient.
        As $\alpha$ increases, the reflected light is concentrated in a narrower region.

        The computational advantage of the Phong model is that if we have normalised
        $\mathbf{r}$ and $\mathbf{v}$ to unit length,
        we can again use the dot product, and the specular term becomes
        \begin{align*}
          I_s = k_s L_s (\mathbf{r}, \mathbf{v})^{\alpha}
        \end{align*}

        \begin{theorem}{Phong Model (with Distance Term)}
          \begin{align*}
            I = \frac{1}{a + bd + cd^2}\bigl(k_d L_d \max(\mathbf{l} \cdot \mathbf{n}, 0)
            + k_s L_s \max\left((\mathbf{r} \cdot \mathbf{v})^{\alpha}, 0\right) \bigr)
            + k_a L_a
          \end{align*}
          That is, distance times diffuse surfaces plus specular surfaces, plus ambient.

          This formula is computed for each light source and for each primary.
        \end{theorem}

      \subsection{The Modified Phong Model}
        If we use the Phong model with specular reflections in our rendering,
        the dot product $\mathbf{r} \cdot \mathbf{v}$ should be recalculated at every point
        on the surface.

        We can obtain an interesting approximation by using the unit vector halfway
        between the viewer vector and the light source vector:
        \begin{align*}
          \mathbf{h} = \frac{\mathbf{l} + \mathbf{h}}{\lvert \mathbf{l} + \mathbf{v} \rvert}
        \end{align*}

        If we replace $\mathbf{r} \cdot \mathbf{v}$ with $\mathbf{n} \cdot \mathbf{h}$,
        we avoid calculation of $\mathbf{r}$.
        We can use the \concept{halfway vector} $\mathbf{h}$ together with the angle $\psi$
        -- the angle between $\mathbf{n}$ and $\mathbf{h}$ to simplify specular calculations.
        When we use the halfway vector in the calculation of the specular term,
        we are using the \concept{Blinn-Phong (or modified Phong) lighting model}.

\end{document}
