\providecommand{\main}{..}
\documentclass[../notes.tex]{subfiles}

\begin{document}
  \setcounter{chapter}{1}
  \chapter{Graphics Programming}
    \section{The Sierpinski Gasket}
      \begin{definition}{Sierpinski Gasket}
        The \concept{Sierpinski Gasket} is an object that can be defined recursively and randomly,
        but has properties that are not random.

        Suppose we start with three points in space.
        As long as the points are not collinear, they are the vertices of a unique triangle
        and also define a unique plane.
        Assume this plane is the plane $z = 0$ and that these points are $(x_1, y_1, 0)$,
        $(x_2, y_2, 0)$ and $(x_3, y_3, 0)$.
        The construction proceeds as:
        \begin{enumerate}
          \item Pick an initial point $\mathbf{p} = (x, y, 0)$ at random inside the triangle.
          \item \label{sierp:2} Select one of the three vertices at random.
          \item Find the point $\mathbf{q}$ halfway between $\mathbf{p}$ and the selected vertex.
          \item Display $\mathbf{q}$ by putting some sort of marker,
            such as a small circle,
            at the corresponding location on the display.
          \item Replace $\mathbf{p}$ with $\mathbf{q}$.
          \item Return to step \ref{sierp:2}.
        \end{enumerate}
      \end{definition}

      \begin{definition}{Immediate Mode Graphics}
        As vertices are generated, they are sent directly to the graphics processor
        for rendering on the display.

        Until recently, was the standard method for displaying graphics.
        A consequence is that there is no memory of the geometric data,
        so if we want to redisplay the scene, we have to go through the entire creation
        and display process again.
      \end{definition}

      \begin{definition}{Retained Mode Graphics}
        Compute all the points first and store them in a data structure.
        Then display all the points through a single function call.

        This approach avoids the overhead of sending small amounts of data to the graphics
        processor for each point we generate,
        at the cost of having to store all the data.
        Because the data are stored, we can redisplay the scene, by resending the stored data
        without having to regenerate it.

        Current GPUs allow us to store the generated data directly on the GPU,
        which avoids the bottleneck caused by transferring the data from the CPU to the GPU
        each time we wish to redisplay the scene.
      \end{definition}

    \section{Programming Two-Dimensional Applications}
      Two-dimensional graphics can be viewed as a special case of three-dimensional graphics.
      The 2D-plane is a subspace of 3D-space.

      A point in the plane $z = 0$ can be represented as $\mathbf{p} = (x, y, 0)$ in the 3D world,
      or as $\mathbf{p} = (x, y)$ in the 2D plane.

      \begin{sidenote}{Vertices}
        A \concept{vertex} is a position in space.
        We use vertices to specify the atomic geometric primitives that are recognized
        by out graphics system.

        The simplest geometric primitive is a \concept{point} in space,
        which is usually specified by a single vertex.
        Two vertices can specify a line segment.
        Three can specify a triangle or circle,
        and so on.
      \end{sidenote}

      \begin{definition}{Clip Coordinate System}
        A cube centred at the origin whose diagonal goes from $(-1, -1, -1)$ to $(1, 1, 1)$.

        Objects outside of this cube will be eliminated (or \concept{clipped}) and cannot appear
        on the display.

        The vertex shader uses transformations to convert geometric data
        specified in some coordinate system to a representation in clip~coordinates,
        and outputs this information to the rasterizer.
      \end{definition}

    \pagebreak

    \section[The WebGL Application Programming Interface] {\sectionmark{The WebGL API}The WebGL Application Programming Interface}
      \sectionmark{The WebGL API}
      Our basic model of a graphics package is a \concept{black box}:
      a system whose properties are described only by its inputs and outputs;
      we know nothing about its internal workings.

      \begin{sidenote}{Function Groups for a Graphics System API}
        $ $\vspace{-1em}
        \begin{descriptenum}[nosep]
          \item[Primitive Functions] Define the low-level objects or atomic entities
            that our system can display.
            WebGL and OpenGL only support points, line segments, and triangles.
          \item[Attribute Functions] Allow us to perform operations to change the
            \concept{attributes} of a primitive.
            Attributes govern the way a primitive appears on the display.
          \item[Viewing Functions] Allow us to specify various views.
            WebGL and OpenGL do not provide any viewing functions,
            but rely on the use of transformations in the application or in the shaders.
          \item[Transformation Functions] Allows users to carry out transformations of objects,
            such as rotation, translation, and scaling.
            In WebGL, we carry out transformations by forming them in our applications,
            and then applying them either in the application or in the shaders.
          \item[Input Functions] Allows us to deal with input devices.
          \item[Control Functions] Enable use to communicate with the window system,
            to initialise our programs, and to deal with any errors that take place
            during the execution of our programs.
          \item[Query Functions] Provides additional information about the operating environment,
            such as camera parameters and values in the frame buffer.
        \end{descriptenum}
      \end{sidenote}

      \subsection{The Graphics Pipeline and State Machines}
        We can think of the entire graphics system as a \concept{state~machine}:
        a black box that contains a finite-state machine.
        The state machine has inputs that come from the application program.
        These inputs may change the state of the machine, or cause the machine to produce
        a visible output.

        One important consequence of the state machine view is that most parameters are
        \emph{persistent}: their values remain unchanged until we explicitly change them
        through functions that alter the state.
        Another consequence is that attributes are part of the state,
        rather than being bound to objects.

        \begin{sidenote}{Shader Functions}
          Part of the old textbook, not in the new.

          OpenGL functions are in a single library called GL.
          Shaders are written in the OpenGL Shading Language (GLSL),
          which has a separate specification.

          Functions that transfer data to the shaders have the following notation:
          \begin{center}
            \mintinline{text}{glSomeFunction*();}
          \end{center}
          where \texttt{*} can be interpreted as $nt$ or $ntv$, where $n$ signifies the number of dimensions,
          $t$ denotes the data type, and $v$ indicates that the variables are specified through
          a pointer to an array, rather than through an argument list.
        \end{sidenote}

      \subsection{Coordinate Systems}
        The advent of \concept{device-independent~graphics} freed application programmers
        from worrying about the details of input and output devices.
        The user's coordinate system become known as the \concept{world~coordinate~system},
        or the \concept{application (or object) coordinate system}.

        Units that the application program uses to specify vertex positions are called
        \concept{vertex~coordinates}, and will generally be the same as object or world coordinates.

        For raster devices, units on the display are called \concept{window coordinates}
        or \concept{screen coordinates}.

        At some point, the values in vertex coordinates must be mapped to window coordinates.
        The graphics system is responsible for this task,
        and the mapping is performed automatically as part of the rendering process.

    \section{Primitives and Attributes}
      We can separate primitives into two classes:
      \begin{descriptimize}
        \item[Geometric Primitives] Specified in the problem domain,
          and include points, line segments, polygons, curves, and surfaces.
          These primitives pass through a geometric pipeline,
          where they are subject to a series of geometric operations that determine
          whether a primitive needs to be clipped,
          where on the display it appears if it is visible,
          and the rasterisation of the primitive into pixels in the framebuffer.

          Because geometric primitives exist in a 2D or 3D space, they can be manipulated
          by operations such as rotation and translation.
          In addition, they can be used as building blocks for other geometric objects
          using these same operations.
        \item[Image (Raster) Primitives] Arrays of pixels.
          Lack geometric properties, and so cannot be manipulated in space in the same way
          as geometric primitives.
          They pass through a separate pipeline on their way to the framebuffer.
      \end{descriptimize}

      The basic WebGL geometric primitives are specified by sets of vertices.
      All WebGL geometric primitives are variants of points, line~segments, and triangular polygons.

      \begin{sidenote}{Specifying Primitives in WebGL}
        $ $\vspace{-1em}
        \begin{descriptimize}
          \item[Points (\mintinline{javascript}{gl.POINTS})] Each vertex is displayed
            at a size of at least one pixel.
          \item[Line Segments (\mintinline{javascript}{gl.LINES})] Successive pairs of vertices
            are interpreted as the endpoints of individual segments.
            Successive segments are usually disconnected, because the vertices are processed
            on a pairwise basis.
          \item[Polylines (\mintinline{javascript}{gl.LINE_STRIP}, \mintinline{javascript}{gl.LINE_LOOP})]
            If successive vertices (and line segments) are to be connected,
            use the line~strip or \concept{polyline} form.
            Many curves can be approximated using this.
            If we wish the polyline to be closed,
            we can locate the final vertex in the same place as the first,
            or use the \mintinline{javascript}{gl.LINE_LOOP} type,
            which will draw a line segment from the final vertex to the first,
            thus creating a closed path.
        \end{descriptimize}

        \begin{itemize}[nosep]
          \item Use \mintinline{javascript}{glPointSize()} to set the current point size in pixels.
          \item Use \mintinline{javascript}{glLineWidth()} to set the current line width in pixels.
        \end{itemize}
      \end{sidenote}

      \subsection{Polygon Basics}
        A \concept{polygon} is an object that has a border that can be described as a line loop,
        but also has a well-defined interior.

        The performance of graphics systems is characterised by the number of polygons
        per second that can be rendered.

        \begin{sidenote}{Properties to Ensure a Polygon is Displayed Correctly}
          $ $\vspace{-1em}
          \begin{descriptimize}[nosep]
            \item[Simple] In 2D, no two edges of a polygon cross each other.
            \item[Convex] All points on the line~segments between any two points
              inside the object, or on its boundary, are inside the object.
            \item[Flat]
          \end{descriptimize}
        \end{sidenote}

        In 3D, polygons present extra difficulties, because all the vertices that
        specify the object are not necessarily in the same plane.
        A property that most graphics systems exploit is that any three vertices
        that are not collinear determine both a triangle,
        and the plane in which that triangle lies.
        Hence, if we use triangles, we can be sure that these objects will be rendered correctly.

        \begin{sidenote}{Drawing Triangles}
          Triangles can be displayed in three ways:
          \begin{itemize}[nosep]
            \item As points corresponding to the vertices using the point type
            \item As edges using one of the line types
            \item With the interiors filled using a triangle type.
          \end{itemize}

          If we want to draw a polygon that is filled and display its edges,
          then we have to render it twice: first as a filled polygon,
          then as a line loop with the same vertices.
          
          \begin{descriptimize}[nosep]
            \item[Triangles (\mintinline{javascript}{gl.TRIANGLES})] The edges are the same
              as they would be if we used line loops.
              Each successive group of three vertices specifies a new triangle.
            \item[Triangle Strip (\mintinline{javascript}{gl.TRIANGLE_STRIP})]
              Based on groups of triangles that share vertices and edges.
              Each additional vertex is combined with the previous two vertices
              two define a new triangle.
            \item[Triangle Fan (\mintinline{javascript}{gl.TRIANGLE_FAN})]
              Based on one fixed point.
              The next two points determine the first triangle,
              and subsequent triangles are formed from one new point,
              the previous point, and the first (fixed) point.
          \end{descriptimize}

          Use \mintinline{javascript}{glPolygonMode()} to tell the renderer to generate
          only the edges or just points for the vertices, instead of fill (the default).
        \end{sidenote}

      \subsection{Triangulation}
        \begin{definition}{Triangulation}
          The process of approximating a general geometric object by subdividing it into
          a set of triangles.
        \end{definition}

        Although every set of vertices can be triangulated, not all triangulations are equivalent.
        Triangulation is a special case of the more general problem of
        \concept{tessellation}, which divides a polygon into a polygonal mesh,
        not all of which need to be triangles.

      \subsection{Text}
        \begin{sidenote}{Types of Text}
          $ $\vspace{-1em}
          \begin{descriptimize}
            \item[Stroke Text] Constructed like other geometric objects.
              Use vertices to define line segments or curves that outline each character.
              If the characters are defined by closed boundaries, we can fill them.
              The advantage of stroke text is that it can be defined to have all the detail
              of any other object, and because it is defined in the same way as other graphical
              objects, it cab be manipulated using standard transformations,
              and viewed like any other graphical primitive.
            \item[Raster Text] Simple and fast.
              Characters are defined as rectangles of bits called \concept{bit blocks}.
              Each block defines a single character by the pattern of 0 and 1 bits
              in the block.
              A raster character can be placed in the framebuffer rapidly
              by a \concept{bit-block-transfer} operation,
              which moves the block of bits using a single function call.

              You can increase the size of raster characters by \concept{replicating}
              or duplicating pixels -- a process that gives larger characters a blocky
              appearance.
              Other transformations of raster characters may not make sense.
          \end{descriptimize}
        \end{sidenote}

      \subsection{Attributes}
        \begin{definition}{Attributes}
          Properties that describe how an object should be rendered.
        \end{definition}

        Available attributes depend on the type of object.
        Attributes may be associated with, or \concept{bound} to, a specific geometric object.
        We can also associate properties with each vertex.

    \section{Colour}
      Our brains receive three values -- the \concept{tristimulus values} --
      that are the responses of the three types of cones to the colour.

      \begin{definition}{Basic Tenet of Colour Theory}
        If two colours produce the same tristimulus values,
        then they are visually indistinguishable.
      \end{definition}

      \begin{definition}{Additive Colour}
        Primary colours add together to give the perceived colour.
        The primaries are usually red, green, and blue.
        Primaries add light to an initially black display, yielding the desired colour.

        An example is the CRT.
      \end{definition}

      \begin{definition}{Subtractive Colour}
        Start with a white surface.
        Coloured pigments remove colour components from light that is striking the surface.
        In these systems, the primaries are usually the \emph{complementary colours}:
        cyan, magenta, and yellow.
      \end{definition}

      We can view a colour as point in a \concept{colour solid}.
      We draw a solid using a coordinate system corresponding to the three primaries.
      The distance along a coordinate axis represents the amount of the corresponding
      primary colour.
      If we normalise the maximum value of each primary to 1,
      we can represent any colour we can produce with this set of primaries as a point
      on a unit cube.

      \subsection{RGB Colour}
        There are two approaches to colour in programming: the \concept{RGB Colour Model},
        and the \concept{Indexed Colour Model}.
        The Indexed model was used previously due to being easier to support in hardware
        because of its lower memory requirements and the limited colours available on displays.

        In a three-primary colour, additive RGB system, there are conceptually
        separate buffers for red, green, and blue images.
        Each pixel might consist of 24~bits: 1~bytes each for red, green, and blue.
        The specification of RGB colours is based on the colour cube.
        Components are specified as numbers between 0 and 1, where where 1 denotes the maximum
        (or \concept{saturated value}) of the corresponding primary.

        RGB can be extended into \concept{RGBA} -- the fourth colour (A, or \concept{alpha})
        is stored in the framebuffer along with the RGB values.
        Alpha is treated as either an opacity or transparency value.
        Transparency and opacity are complements of each other:
        an \emph{opaque} object lets no light through it, a \emph{transparent} object
        passes all light through it.
        Opacity values can range from fully transparent ($A = 0$) to fully opaque ($A = 1$).

        In WebGL, unlike desktop OpenGL, if blending is not enabled,
        the value of A multiplies the R, G, and B values.
        Thus, values of A less than 1.0 mute the colours.

      \subsection{Indexed Colour}
        Early graphics systems had framebuffers that were limited in depth.
        We could divide each pixel's limited 8~bits into smaller groups of bits
        and assign red, green, and blue to each.
        This technique did not give flexibility in colour assignment.
        Indexed colour provided a solution that allowed applications to display a wide range
        of colours as long as the application did not need more colours than could be
        referenced.

        Choose for each application a limited number of colours from a large selection
        (our \concept{palette}).
        We can select colours by interpreting out limited-depth pixels as indices into a table
        of colours rather than as colour values.
        If our framebuffer has $k$ bits per pixel, then each pixel value or index is an integer
        between $0$ and $2^k - 1$.
        Suppose we can display each colour component with a precision of $m$~bits:
        we can choose from $2^m$~reds, $2^m$~greens, and $2^m$~blues.
        So, we can produce any $2^{3m}$ colours on the display,
        but the framebuffer can specify only $2^k$ of them.
        Handle the specification through a user-defined \concept{colour~lookup~table}
        that is of size $2^k \times 3m$.
        The user program fills the $2^k$ entries of the table with the desired colours,
        using $m$ bits for each of red, green, and blue.
        Once the user has constructed the table, they can specify a colour by its index,
        which points to the appropriate entry in the loop table.

        Historically, color index mode was important because it required less memory
        for the framebuffer and fewer other hardware components.
        Cost is no longer an issue, and this mode provides a few problems:
        \begin{itemize}
          \item When we work with dynamic images that must be shaded,
            usually we need more images than are provided by the color-index mode.
          \item The interaction with the window system is more complex than with RGB color.
        \end{itemize}

    \section{Viewing}
      The specification of objects in out scene is completely independent of our specification
      of the camera.
      Once we have specified both the scene and the camera, we can compose an image.

      The simplest view, and WebGL's default is the \concept{orthographic projection}.
      All projectors are parallel and the centre of projection is replaced by a
      \concept{direction of projection}.
      Projectors are perpendicular to the projection plane.
      The orthographic projection takes a point $(x, y, z)$ and projects it into the point
      $(x, y, 0)$.

    \section{Control Functions}
      \subsection{Interaction with the Window System}
        A \concept{window} is a rectangular area of a display.
        It has a height and width, and because the window displays the contents of the framebuffer,
        positions in the window are measured in \concept{window (or screen) coordinates}
        where the units are pixels.
        Window coordinates are 3D, whereas screen coordinates are 2D.
        The term \concept{window~system} refers to the multiwindow environment provided
        by different systems.

        References to positions in a window are relative to one corner of the window,
        normally the top left.
        WebGL functions assume that the origin is the bottom left.

      \subsection{Aspect Ratio and Viewports}
        The \concept{aspect~ratio} of a rectangle is the ratio of the rectangle's
        width to its height.
        If the aspect ratio of the viewing rectangle is not the same as the aspect ratio
        of the specified window, then objects are distorted on the screen.
        We can avoid this distortion by ensuring the clipping rectangle and the display window
        have the same ratio.

        A \concept{viewport} is a rectangular area of the display window.
        By default, it is the entire window, but it can be set to any smaller size in pixels
        using the function
        \begin{minted}[autogobble]{javascript}
          gl.viewport(x, y, w, h);
        \end{minted}
        where \mintinline{javascript}{(x, y)} is the lower-left corner of the viewport,
        and \texttt{w} and \texttt{h} give the width and height, respectively.

      \subsection{Application Organisation}
        The mechanism used for controlling graphics employed by most graphics and window systems
        is to use \concept{event~processing}, which gives us interactive control in our programs.
        \concept{Events} are changes that are detected by the operating system.
        An event may generate data that is stored with the occurrence of the event.
        When events occur, they are placed in a queue, the \concept{event~queue},
        which can be examined by an application program or by the OS.

        With WebGL, we will be able to identify the events to which we want to react through
        functions called \concept{event~listeners} or \concept{callbacks}.
        A callback function is associated with a specific type of event.

    \rulechapterend

\end{document}
